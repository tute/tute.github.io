
Me parece casi necesario viajar, seguramente, porque en cada viaje crec\'i un
poquito, evolucion\'e como persona, pens\'e cosas que nunca --en la vida
diaria-- pienso. Vuelvo cuestionando cosas, celebrando otras, con ganas de
empezar algo nuevo y de dejar algo viejo; en fin, con ganas de mejorar.

Encontr\'e que en la vida curricular no cuento con la simpleza y apertura
mental que tengo de viaje, lo atribuyo a dos cosas.

La primera es que estar de viaje implica un corte en mi rutina, dejando todo
el tiempo libre para ocupar el cerebro en lo que quiera, sin obligaciones
pendientes. Estoy abierto a pensamientos que en la rutina juzgar\'ia como
``p\'erdidas de tiempo'', y que sin embargo son m\'as profundos y de largo
plazo que los que tengo durante el a\~no.

La segunda raz\'on es que los tiempos de fin de a\~no (que es cuando
generalmente tengo vacaciones) se prestan para estas evaluaciones: qu\'e se
logr\'o en el a\~no pasado, cu\'anto se lo disfrut\'o, qu\'e ser\'ia bueno
hacer (o dejar de hacer) en el a\~no entrante. Es m\'as f\'acil de que surjan
esos pensamientos en estas \'epocas del a\~no, mirar si el camino recorrido se
acerca a lo que queremos y creemos bueno, o hay que ajustarlo un poco.

En cambio, ocuparme de estos cuestionamientos ``de fondo'' durante el a\~no,
me alejan de las metas m\'as cortitas y cercanas (aprobar el examen que se
acerca, terminar el trabajo a tiempo). M\'as vale terminar las tareas que
estoy haciendo, y, una vez concluidas, hacer un entretiempo para evaluarlas.
As\'i, creo, me acerco a un equilibrio, y voy proponiendo peque\~nas y grandes
metas a mi vida, e intento seguirlas.

\textexclamdown Pensar que en un principio cre\'ia que viajar se trataba de
divertirse y de mejorar mi entrenamiento en bici! La verdad es que la
experiencia result\'o siempre mucho m\'as espiritual.

Por eso me parece tan importante salir a conocer lugares. Descubro ahora que,
en gran medida, por eso amo viajar.\\

Eugenio Costa.

26 de Diciembre de 2007